\begin{resume}
L'objet de cette th\`ese est le d\'eveloppement et l'\'etude d'algorithmes
de planification de mouvements optimaux pour des syst\`emes
anthropomorphes sous-actionn\'es et hautement dimensionn\'es, \`a l'instar
des robots humano\"ides et des acteurs virtuels. Des m\'ethodes de
planification al\'eatoires et de commande optimale sont propos\'ees et
discut\'ees. Une premi\`ere contribution concerne l'utilisation d'une
m\'ethode efficace de recherche dans un graphe pour l'optimisation de
trajectoires de marche planifi\'ees pour un syst\`eme mod\'elis\'e par sa
boîte englobante. La deuxi\`eme contribution concerne l'utilisation de
m\'ethodes de planification al\'eatoires sous contraintes afin de
planifier de façon g\'en\'erique des mouvements corps-complet de marche et
manipulation. Enfin nous d\'eveloppons une approche algorithmique qui
combine des m\'ethodes de planification al\'eatoires sous contrainte et de
commande optimale; cette approche permet de g\'en\'erer des mouvements
dynamiques, rapides, et sans collision, en pr\'esence d'obstacles dans
l\'environnement du syst\`eme.
\end{resume}
