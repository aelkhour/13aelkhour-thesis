\chapter*{Remerciements}\label{chap:merci}

\epigraph{\begin{CJK*}{UTF8}{min}%
%\ruby{石}{いし}の\ruby{上}{うえ}にも\ruby{三}{さん}\ruby{年}{ねん}%
石の上にも三年%
\end{CJK*}}{Proverbe japonais}

\vspace{1cm}

J'aimerais tout d'abord remercier David et Théo de bien avoir voulu
relire mon manuscrit et de m'avoir fait part de leurs remarques. Je
remercie également Rodolphe Gelin d'avoir accepté de faire partie
de mon jury de thèse.


Cette thèse est l'aboutissement d'un parcours académique où de
nombreux chemins de travers auront été empruntés. Notamment, je
remercie toute l'équipe du LRDE et en particulier Akim et Théo de
m'avoir initié à la recherche. Sans eux, il m'est évident que ces
dernières années auraient eu une autre couleur. Une pensée
particulière également pour mes collègues de GOSTAI et à
Jean-Christophe Baillie pour m'avoir amené à la robotique et, à son
corps défendant~(!), à cette thèse. Jean-Paul a ensuite su trouver les
mots pour, en quelques heures, expliquer ce qui me paraissait alors
d'une complexité insondable. Arrivé à Tsukuba, Florent, Pierre-Brice et
Olivier ont pris le temps de m'expliquer la planification de mouvement
et l'optimisation numérique, bien au-delà de leur rôle
d'encadrants. Un second merci à Florent qui est ensuite devenu mon
directeur de thèse et qui possède cette faculté particulière de
pouvoir transformer ce qui apparaît comme trois années difficiles en
une expérience vraiment plaisante. Ma semaine au sein du groupe
LAGADIC fut courte, mais enrichissante! Merci à Eric Marchand pour
l'introduction courte, et efficace, à la vision par ordinateur. Merci
à Claire Dune pour ses conseils et sa disponibilité. Enfin, je
remercie Eiichi et Abder pour les précieuses discussions que nous
avons pu avoir, et qui risquent de continuer encore un peu!


Le JRL et Gepetto ont été deux groupes extraordinaires en terme de
travail et d'ambiance. Un grand merci à tous les permanents pour leur
bienveillance. Quant aux doctorants, merci au bureau des mèmes -
Antonio, Maxime et Samory - en particulier, mais aussi à: Paul,
François, Pierre, Karim, Mehdi, Nicolas, Nosan, Oussama, David, Sovan,
Manish, Séb, Ali, Valentin, Duong et Layale. Mention spéciale aux
jolis Gepettistes de c\oe ur: Wassima et Wassim.


Ma gratitude revient à Yoshiko notre chère professeur qui nous a
enseigné le japonais sans jamais se décourager! Merci à Ayaka de
prendre le temps de déchiffrer mes lettres en japonais, à Alice
d'agiter les bras en l'air sans raison, à Martin pour les photos qu'il
me donnera peut-être un jour, à Pauline d'avoir les yeux qui brillent
à chaque fois qu'on lui montre un kanji, à Antonio et Maya pour leur
future invitation au Liban, à Mélanie pour avoir formé le meilleur
binôme de canoë que l'on puisse imaginer, à Marine pour sa
perspicacité et ses blagues, à Yuri pour cette incroyable semaine en
Italie, à Julie et Anaïs pour les séances de psychologie de groupe, à
Rika, Rie, Tomoko, Haruka, Takahiro, Waritta, Yukari, Kei, Keiko et
Mariko. À~Yukiko pour m'avoir accompagné durant cette dernière année
et pour son aide à tous les niveaux\ldots Mes collègues se joignent à
moi pour la remercier de son attention constante quant au niveau de
glycémie du groupe.


\begin{CJK*}{UTF8}{min}%
À~mes~バカ~préférées, Yuka et Marion, pour ce noël inoubliable et tant
d'autres choses encore.
\end{CJK*}



À ma famille, à mes frères, et à mes parents qui ont poussé la
dévotion pour leurs enfants à l'extrême limite de leurs capacités.
