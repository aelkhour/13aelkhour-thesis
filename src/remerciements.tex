\chapter*{Remerciements}

J'ai profit\'e d'un accueil exceptionnel au LAAS-CNRS durant mes trois
ann\'ees de th\`ese. C'est pourquoi je tiens tout d'abord \`a
remercier ses directeurs successifs Raja Chatila, Jean-Louis Sanchez
et Jean Arlat, le directeur du th\`eme Robotique Rachid Alami, et plus
sp\'ecifiquement les directeurs successifs du groupe Gepetto Jean-Paul
Laumond et Philippe Sou\`eres.

Je tiens \`a exprimer toute ma gratitude envers mes directeurs de
th\`ese Florent Lamiraux et Michel Ta\"ix. Leur confiance, leurs
connaissances approfondies, leurs pr\'ecieux conseils et leur
sympathie ont directement contribu\'e au travaux que j'ai effectu\'es
et au plaisir que j'en ai tir\'e.

C'est un honneur d'avoir Maren Bennewitz et Abderrahmane Kheddar comme
rapporteurs de ma th\`ese, et je les remercie sinc\`erement pour leur
relecture attentive de mon manuscrit. Je remercie \'egalement Brigitte
d'Andr\'ea-Novel, Timothy Bretl, Patrick Dan\`es et Rodolphe Gelin
d'avoir accept\'e de faire partie de mon jury de th\`ese, ainsi que
pour leurs remarques et discussions int\'eressantes.

J'ai eu la chance d'effectuer un s\'ejour scientifique \`a
l'Universit\'e de Heidelberg dans le groupe ORB . Je tiens \`a
remercier Katja Mombaur pour ses pr\'ecieux conseils et l'excellent
accueil qu'elle m'a r\'eserv\'ee.

J'ai \'et\'e tr\`es marqu\'e par l'esprit d'\'equipe qui r\`egne dans
le groupe Gepetto, et esp\`ere y avoir contribu\'e durant ces trois
ann\'ees. Je remercie chaleureusement Nicolas Mansard et Olivier
Stasse qui, sans \^etre directement impliqu\'es dans mes travaux,
m'ont fourni tout le soutien scientifique et technique dont un
doctorant pourrait r\^ever.

\bigskip

Ces trois derni\`eres ann\'ees sont pass\'ees rapidement; ceci est
principalement d\^u \`a mon c\^otoiement au quotidien de doctorants et
stagiaires sympathiques et brillants, dont la bonne humeur a ajout\'e
encore plus de plaisir \`a ces travaux de recherche. J'ai eu la joie
de collaborer \'etroitement avec S\'ebastien Dalibard, Martin Felis,
David Flavign\'e et Thomas Moulard; je les remercie pour leur
d\'evouement \`a nos travaux ainsi que pour leur amiti\'e.

Je me sens privil\'egi\'e d'avoir pu rencontrer de glorieux anciens,
qui m'ont inculqu\'e les pr\'eceptes de l'esprit d'\'equipe et de la
bonne ambiance. Merci donc \`a Duong, Fran\c{c}ois, Layale, Manish,
Maxime, Nicolas, Oussama, Samory, Sovan, Valentin, Wassim, et
Wassima. Je remercie \'egalement tous les actuels membres qui
perp\'etuent la tradition: Aiva, Andreas, Arturo, Francesco, He,
Henning, L\'eo, Laurent, Mauricio, Mehdi, Olivier, Oscar, Perle, avec
une mention sp\'eciale pour Justin Carpentier, Olivier Roussel, et
Jorrit T'Hooft qui ont consacr\'e une semaine de leur temps \`a la
construction d'un magnifique mur de brique. Je leur souhaite \`a tous
bonne continuation.

S'il arrive un jour \`a lire et comprendre ce manuscrit, je tiens \`a
remercier le robot HRP-2 14 pour avoir support\'e, sans jamais se
plaindre, les collisions, les chutes, et tous les mouvements
``inhumains'' que je lui ai fait faire.

Je souhaite exprimer toute ma reconnaissance \`a ma famille, et plus
particuli\`erement \`a mes parents et mon fr\`ere, pour leur amour,
leurs conseils attentionn\'es et leur soutien constant.

Enfin je souhaite remercier Maya pour son soutien, son amour, et pour
m'avoir accompagn\'e, malgr\'e la distance, durant ces trois
derni\`eres ann\'ees qui ont men\'e jusqu'\`a ma soutenance. Et la fin
n'est que le d\'ebut.

\begin{verbatim}
(0,0)
/)_)
 ""
\end{verbatim}
